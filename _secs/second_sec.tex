\section{ইতিহাস}
বাংলা ভাষার ইতিহাসকে সাধারণত তিন ভাগে ভাগ করা হয়

\begin{enumerate}
    \item প্রাচীন বাংলা
    \item মধ্য বাংলা
    \item আধুনিক বাংলা
\end{enumerate}

বাংলায় প্রায় ৭৫,০০০ পৃথক শব্দ রয়েছে, যার মধ্যে

\begin{itemize}
    \item ৫০,২৫০ (৬৭\%) শব্দ "তৎসম" (সংস্কৃত ভাষা থেকে সরাসরি গৃহীত);
    \item ২১,০০০ (২৮\%) শব্দ "তদ্ভব" (বাংলা ভাষার শব্দসমূহ যার উৎস পালি এবং প্রাকৃত ভাষা থেকে হয়েছে);
    \item প্রায় ৫০০০ টি বিদেশি শব্দ
\end{itemize}